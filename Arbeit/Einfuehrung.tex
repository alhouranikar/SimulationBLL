Häufig werden die Euler-Gleichungen aufgrund ihrer einfacheren Handhabung statt der vollständigen Navier-Stokes-Gleichungen verwendet, dazu zählen Anwendungen wie die Untersuchung von Tragflächenprofilen im Flugzeugbau, Untersuchungen zum Blutfluss durch den Körper in der Medizin oder die Modellierung von Rohrströmungen in technischen Bereichen. Da die Euler-Gleichungen ein Spezialfall der Navier-Stokes-Gleichungen sind, sind sie jedoch in guter Näherung nur für laminare, nicht viskose, inkompressible Strömungen anwendbar. Daher finden sich nicht so viele Anwendungsbereiche wie bei den vollständigen Navier-Stokes-Gleichungen. Diese Arbeit soll der Untersuchung des Zusammenhangs zwischen analytischen Lösungsmethoden und der numerischen Berechnung der Euler-Gleichungen gewidmet sein. Im Zuge dessen wird eine eigens erstellte Simulation verwendet, um entsprechend valide numerische Ergebnisse zu erhalten.

\subsection{Was beschreiben die Navier-Stokes-Gleichungen?}

\subsection{Historie}

\subsection{Herausforderungen in der Untersuchung der Navier-Stokes-Gleichungen}

\subsection{Anwendungen in der Praxis}