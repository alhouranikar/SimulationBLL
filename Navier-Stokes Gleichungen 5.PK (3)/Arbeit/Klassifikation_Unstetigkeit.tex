\subsubsection{Klassifizierung partieller Differentialgleichungen}

Buch~\cite[S. 115]{epstein_partial_2017} nennt folgende allgemeine Form partieller Differentialgleichungen zweiter Ordnung:
\begin{align}
\label{zeile1}
    a u_{xx} + 2b u_{xy} + c u_{yy} = d
\end{align}
bzw. im quasi-linearen Fall:
\begin{align*}
    a(x,y,u,u_x,u_y) u_{xx} + 2b(x,y,u,u_x,u_y) u_{xy} + c(x,y,u,u_x,u_y) u_{yy} = d(x,y,u,u_x,u_y).
\end{align*}
Gegebene Anfangsbedingungen der Form 
\begin{align*}
    u=\hat{u}(r), \quad u_x = \hat{u}_x (r), \quad u_y = \hat{u}_y (r)
\end{align*}
(mit dem Kurvenparameter $r$) - welche für die Lösung von Gleichungen dieser Form unerlässlich sind - müssen offensichtlich die Gleichung selbst erfüllen. Außerdem muss nach der Kettenregel und der Bedingung, dass alle Ableitungen an jedem Punkt der gegebenen Kurve $r$ bzw. $\hat{u}(r)$ definiert sein müssen, gelten:
\begin{align}
\label{zeile2}
    \diff{\hat{u}_x}{r} &= u_{xx} \diff{\hat{x}}{r} + u_{xy} \diff{\hat{y}}{r} \\ 
\label{zeile3}
    \diff{\hat{u}_y}{r} &= u_{yy} \diff{\hat{y}}{r} + u_{xy} \diff{\hat{x}}{r}.
\end{align}
Werden nun die Gleichungen ~\eqref{zeile1}, ~\eqref{zeile2} und ~\eqref{zeile3}, welche von den Anfangsbedingungen erfüllt werden müssen, in ein Gleichungssystem kombiniert, ergibt sich:
\begin{align}
\label{lgs_qual}
% Verschönern!!!
    \renewcommand{\arraystretch}{1.15}
    \begin{bmatrix}
        a & 2b & c \\
        \diff{\hat{x}}{r} & \diff{\hat{y}}{r} & 0 \\
        0 &  \diff{\hat{x}}{r} & \diff{\hat{y}}{r} \\
    \end{bmatrix}
    \begin{bmatrix}
        u_{xx} \\
        u_{xy} \\
        u_{yy} \\
    \end{bmatrix}
    =
    \begin{bmatrix}
        d \\
        \diff{\hat{u}_x}{r} \\
        \diff{\hat{u}_y}{r}
    \end{bmatrix}
\end{align}
damit dieses Gleichungssystem eindeutig lösbar ist, muss die Determinante der Koeffizientenmatrix ungleich null sein. Nach der \emph{Regel von Sarrus} ergibt sich für Gleichung ~\eqref{lgs_qual}:
\begin{align}
\label{determinante}
\determ{A} = a\cdot \diff{\hat{y}}{r} \cdot \diff{\hat{y}}{r} + 0 + c \cdot \diff{\hat{x}}{r}\cdot \diff{\hat{x}}{r} - 2b \cdot \diff{\hat{x}}{r} \cdot \diff{\hat{y}}{r} - 0 - 0.
\end{align}
Wäre das Gleichungssystem an jeder Position auf der Anfangskurve $r(x,y)$ eindeutig lösbar, sind diese Anfangsbedingungen nicht ausreichend, um die partielle Differentialgleichung zu lösen, da sie an keiner Stelle vorschreiben, wie sich die Lösungsfunktion außerhalb dieser gegebenen Kurve verhalten. Es lässt sich zeigen, \begin{comment}
    ~\cite{Epstein}
\end{comment}
dass gerade an den Stellen, an denen das Gleichungssystem nicht eindeutig lösbar ist, Charakteristiken (eine Definition soll hier nicht genannt werden) aufweist. Diese sind essenziell für die Lösungstheorie partieller Differentialgleichungen und bestimmen die Art und Weise partielle Differentialgleichungen zu lösen. Setzen wir ~\eqref{determinante} gleich null, und multiplizieren beide Seiten mit $\left(\diff{r}{\hat{x}}\right)^2$, erhalten wir eine quadratische Gleichung, deren Lösungen mit der \emph{pq-Formel} gefunden werden können:
\begin{align}
\label{quad_gleichung_det}
    \lr{(}{)}{\diff{\hat{y}}{\hat{x}}}_{1,2} = \frac{b}{a} \pm \sqrt{\lr{(}{)}{\frac{b}{a}}^{2} - \frac{c}{a}}.
\end{align}
Für die Diskriminante $D$ der Gleichung können folgende Fälle eintreten:
\begin{align}
    x_{1,2} = 
    \begin{cases}
        D>0: \quad &x_1 \neq x_2  \land x_1, x_2 \in \mathbb{R} \\
        D=0: \quad &x_1 = x_2 \land x_1, x_2 \in \mathbb{R} \\
        D<0: \quad &x_1 \neq x_2 \land x_1, x_2 \in \mathbb{C} \\
    \end{cases}
\end{align}
\begin{Definitionbox}[]
    Eine partielle Differentialgleichung zweiter Ordnung ist: \\
    \textbf{hyperbolisch}, wenn es zwei unterschiedliche Lösungen ($\in \mathbb{R}$) der Gleichung ~\eqref{quad_gleichung_det} gibt, \\
    \textbf{parabolisch}, wenn es eine Lösung ($\in \mathbb{R}$) gibt, \\
    \textbf{elliptisch}, wenn es zwei unterschiedliche Lösungen ($\in \mathbb{C}$) gibt.
\end{Definitionbox}

Um die Euler-Gleichungen nun genauer zu betrachten, soll ihre Fourier-Transformierte berechnet werden, um den Umgang mit dieser Gleichung zu vereinfachen.
Dazu soll die Euler-Gleichung in folgendes Gleichungssystem umgeschrieben werden:
\begin{align}
\label{eq:euler_system}
    &\begin{bmatrix}
        0 & 0 & 0 \\
        0 & \rho  & 0 \\
        0 & 0 & \rho \\
        0 & u_1   & u_2 \\
    \end{bmatrix}
    \begin{bmatrix}
        \rho_t \\
        u_{1_t} \\
        u_{2_t} \\
    \end{bmatrix}
    +
    \begin{bmatrix}
        \rho & 0 & 0 \\
        2\rho u_1 & 0 & 1 \\
        \rho u_2 & \rho u_1 & 0 \\
        \rho e + p + u_1 & u_2 & u_1 \\
    \end{bmatrix}
    \begin{bmatrix}
        u_{1_x} \\
        u_{2_x} \\
        p_x \\
    \end{bmatrix} \\
    &+
    \begin{bmatrix}
    \rho             & 0 & 0 \\
    \rho u_1         & \rho u_2 & 0 \\
    2\rho u_2        & 0 & 1 \\
    \rho e + p + u_2 & u_1 & u_2 \\
    \end{bmatrix}
    \begin{bmatrix}
        u_{2_y} \\
        u_{1_y} \\
        p_y \\
    \end{bmatrix}
    =
    \begin{bmatrix}
        0 \\
        \rho g_1 \\
        \rho g_2 \\
        \rho (u_1 g_1 + u_2 g_2)\\
    \end{bmatrix}
\end{align}
Für die reine Klassifizierung der Euler-Gleichungen werden folgende Vereinfachungen angenommen, um die Fourier-Transformation zu ermöglichen und damit die Analyse zu simplifizieren: es existiere eine Lösung $u$ der Euler-Gleichung, welche sowohl stetig als auch integrierbar ist; der Einfluss externer Kräfte (z.B. der Gravitation) wird vernachlässigt, damit die Funktion $u(x,y,t)$ in allen unabhängigen Variablen im Unendlichen gegen null strebt (\cite[nach][S. 802]{bronstejn_taschenbuch_2020}).


% HIER DEN REST ENTFERNEN
Damit solche Charakteristiken nun existieren, muss, wie im Fall von partiellen Differentialgleichungen zweiter Ordnung, für festgelegte Anfangsbedingungen das ursprüngliche Gleichungssystem erfüllt sein sowie die Kettenregel der Differentiation gelten. Unter diesen Einschränkungen ergibt sich für die Euler-Gleichung~\eqref{eq:euler} folgendes System:
\begin{align*}
    fda
\end{align*}
Ein Ansatz zur Gewinnung der Charakteristiken ist die Lösung der Eigenwert-Gleichung für charakteristische Kurven, wobei diese durch die Gleichungen
\begin{align}
    \diff{x}{s} = a(x,y,u) \qquad \diff{y}{s}=b(x,y,u) \qquad \diff{u}{s} = c(x,y,u)
\end{align}
gegeben sind. Diese Charakteristiken sind zu integrieren und offenbaren die Lösung der partiellen Differentialgleichung, da
\begin{align*}
    \diff{x}{s} \pd{u}{x} + \diff{y}{s} \pd{u}{y} = \diff{u}{s}
\end{align*}
ist. Unter Angabe geeigneter Anfangsbedingungen kann diese nun gewöhnliche Differentialgleichung über dem Kurvenparameter $s$ gelöst werden. Da im Fall von einem Gleichungssystem partieller Differentialgleichungen der Form
\begin{align*}
    \boldsymbol{Au_x} + \boldsymbol{Bu_y} + \boldsymbol{Cu_t} = \boldsymbol{D}
\end{align*}
(wie in Gleichung~\eqref{eq:euler_system}) die Koeffizienten der Ableitungen durch Matrizen gegeben sind, ergibt sich:
\begin{align*}
    \boldsymbol{A}-\boldsymbol{B}\diff{x}{y} - \boldsymbol{C}\diff{x}{t} = 0
\end{align*}
als Eigenwertgleichung für die Charakteristiken des Systems.

\subsubsection{Entstehung von Unstetigkeiten}

Sei eine parametrisierte Lösungsoberfläche $(r,v,s)$ gegeben. Da sie durch Parameter anstatt Variablen (von denen die Funktion $u$ abhängig ist) dargestellt wird, ist ihre Eindeutigkeit nicht definiert. Nur wenn die Charakteristiken auch auf die Oberfläche $(x,y,t)$ durch eine bijektive Funktion abgebildet werden können. Vor diesem Hintergrund sei die Variablentransformation
\begin{align*}
    \pd{(r,v,s)}{(x,y,t)}
\end{align*}
mithilfe der \emph{Jacobi-Matrix}
\begin{align}
\label{eq:jacobi}
    \pd{(r,v,s)}{(x,y,t)} := \begin{bmatrix}
        \pd{r}{x} & \pd{r}{y} & \pd{r}{t} \\
        \pd{v}{x} & \pd{v}{y} & \pd{v}{t} \\
        \pd{s}{x} & \pd{s}{y} & \pd{s}{t} \\
    \end{bmatrix}
\end{align}
definiert. Ist die Determinante der Matrix für eine Stelle $(r_0 ,v_0, s_0)$ gleich null, bedeutet dies, dass die Variablentransformation nicht eindeutig ist. Im Kontext der charakteristischen Oberflächen ist also eine Abbildung auf das ursprüngliche Koordinatensystem $(x,y,t)$ nicht möglich, die erhaltene Lösung erlaubt also eine Unstetigkeit an dieser Stelle. Anschaulich gesprochen \glqq faltet \grqq sich die Lösungsoberfläche.
Darauf basierend, bietet die \emph{Rankine-Hugoniot-Bedingung} eine Kondition für die Ausbreitung von solchen Unstetigkeiten, welche nun direkt am Beispiel der Euler-Gleichung~\eqref{eq:euler} präsentiert werden soll.
Zunächst soll die Gleichung~\eqref{eq:euler} in Abhängigkeit von $T$ formuliert werden:
\begin{align*}
    \pd{}{t}T + \pd{}{x}X(T) + \pd{}{y}Y(T) = Q(T).
\end{align*}

\begin{Definitionbox}
    Die Rankine-Hugoniot-Bedingung (RH-Bedingung) stellt sich als folgende Voraussetzung für die Ausbreitung von Unstetigkeiten dar~\cite[Vgl.][S. 18]{vides_simple_2014}:
    \begin{align}
    \label{eq:HN_bedingung}
        s_p = \frac{n_x u_1 + n_y u_1}{-p},
    \end{align}
    wobei $\vec{n} = (n_x, n_y)^T$ der 2-dimensionale Normalenvektor einer Unstetigkeitskurve ist.
\end{Definitionbox}
Jene Forderung ist intuitiv, da der Normalenvektor einer Kurve, auf welcher die Funktionen $u_1 , u_2 , p$ nicht stetig sind, zu dem Gradienten auf dieser Kurve senkrecht stehen muss (da der Gradient gerade hier unendlich groß ist).