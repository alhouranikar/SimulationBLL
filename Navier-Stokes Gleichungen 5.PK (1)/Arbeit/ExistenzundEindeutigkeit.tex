Um die Eindeutigkeit bzw. grundlegende Existenz von Lösungen für partielle Differentialgleichungen zu beweisen, ist es in der physikalischen Realität - im Gegensatz zu den idealisierten Bedingungen der Mathematik - häufig sinnvoll, statt vollständige Lösungen für genannte Gleichungen zu finden, schwache Lösungen zu berechnen.
\begin{Definition}
    Sei die $n$-dimensionale partielle Differentialgleichung
    \begin{align}
    \label{PDE_Existenz}
        a u_{xx} + 2b u_{xy} + c u_{yy} = d
    \end{align}
    gegeben. Eine Funktion $u\in \mathbb{R}$ ist eine \emph{schwache Lösung} von Gleichung~\eqref{PDE_Existenz} auf der Menge $A$, wenn für $\forall \varphi \in C_0 ^{\infty} (A)$ folgende Bedingung erfüllt ist:
    \begin{align*}
        \int_{A}{\varphi (a u_{xx} + 2b u_{xy} + c u_{yy}) \dfr x} = \int_{A}{\varphi d \dfr x}
    \end{align*}
    bzw. mit der Umformung in Abschnitt~\ref{sec:anhang_rechnung_1}: % HIER VERBESSERN, dass der gesamte Unterabschnitt ausgeschrieben wird
    \begin{align*}
        % HIER UMFORMUNG EINFÜGEN
    \end{align*}
\end{Definition}
Mithilfe dieser schwachen Lösungsvorgabe lässt sich einsehen, dass eine entsprechende Lösung im Sobolev-Raum liegen muss.