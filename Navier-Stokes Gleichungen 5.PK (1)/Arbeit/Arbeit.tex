\documentclass[11pt,a4paper]{article}
\usepackage[ngerman]{babel}
\usepackage[utf8]{inputenc}
\usepackage[T1]{fontenc} 
\usepackage{csquotes}
\usepackage{blindtext}
\usepackage{amssymb}
\usepackage{amsmath}
\usepackage{hyperref}
\usepackage{enumitem}
\usepackage{verbatim} % Kommentare
\usepackage{enumitem} % itemize-Umgebungen verschönern
\usepackage{graphicx} % für \includegraphics
\usepackage{anyfontsize} % verschiedene Schriftgrößen für Abbildungen
\usepackage{mathrsfs} % für geschwungene Schriftart
\usepackage{amsthm} % für normalen Text innerhalb von Definitionen
\usepackage{biblatex} % für modernere Weise des Zitierens
\usepackage{xcolor} % für importieren von Inkscape-Dateien
\usepackage{float} % damit Tabellen festen Ort haben können
%\usepackage{classicthesis}

\usepackage[a4paper, left=4cm, right=2cm]{geometry}
\usepackage{fontspec} %\setmainfont
%\setmainfont{Times New Roman}
\usepackage{ragged2e} %Blocksatz
\usepackage{scrlayer-scrpage}
\clearpairofpagestyles
\ihead{Karam Alhourani, Linus Lungwitz}
%\chead{Navier-Stokes-Gleichungen}
\ohead{\sectionautorefname}
\cfoot{Seite \thepage}
\usepackage{setspace}
\spacing{1.5}

\newcommand{\pd}[2]{\frac{\partial #1}{\partial #2}} % partielle Ableitung
\newcommand{\pds}[2]{\frac{\partial^2 #1}{\partial #2^2}} % zweite partielle Ableitung nach einer Variable
\newcommand{\pdsu}[3]{\frac{\partial^2 #1}{\partial #2 \partial #3}} % zweite partielle Ableitung nach zwei Variablen
\newcommand{\diff}[2]{\frac{d #1}{d #2}} % normale Ableitung
\newcommand{\determ}[1]{\text{det}\,\left(#1\right)} % Determinante
\newcommand{\lr}[3]{\left#1 #3 \right#2} % left-right-Umgebung
\newcommand{\dfr}{\text{d}} % Differential für das Ende von Integralen
\newcommand{\supp}{\text{supp}} % Träger einer Funktion

\theoremstyle{definition}
\newtheorem{Definition}{Definition}

\numberwithin{equation}{section} % Nummerierung der Gleichungen mit Kapiteln
\counterwithin{figure}{section} % Nummerierung der Bilder mit Kapiteln

\allowdisplaybreaks % Erlauben von Seitenumbrüchen innerhalb von align

\addbibresource{Arbeit/Literatur_5PK.bib} % Literaturverzeichnis

\setcounter{secnumdepth}{3}

\author{Karam Alhourani, Linus Lungwitz}
\title{Navier-Stokes-Gleichungen}
\begin{document}
\maketitle

\newpage

\tableofcontents

\newpage

\section*{Notation}

Folgende Notation wird in der Arbeit verwendet. Abweichungen werden an entsprechenden Textstellen markiert.

\begin{itemize}[label={}]
    \item $\boldsymbol{E}$ - Einheitsmatrix (wenn nicht anders angegeben in 2 Dimensionen) \\
    \item $x^T$ - transponierter Vektor zum Vektor x (analog für eine Matrix $A$) \\
    \item $f_x$ - partielle Ableitung der Funktion $f$ nach der Variable $x$ \\
    \item $f^{(n)}$ - n-te partielle Ableitung der Funktion $f$ (die Variable ist als solche gekennzeichnet) \\
    \item $\determ{A}$ - Determinante der Matrix $A$ \\
    \item $x_{1,2}$ - Nullstellen einer Funktion \\
    \item $\mathcal{O}(x)$ - Term der Größenordnung $x$ \\
    \item $\left.\pd{f(x,y)}{x}\right\vert_{i,j}$ - Ableitung der Funktion $f(x,y)$ an der Stelle $i,j$ \\
    \item $[a,b]$ - geschlossenes Intervall von $a$ bis $b$ \\
    \item $]a,b[$ - offenes Intervall von $a$ bis $b$ \\
    \item $\mathscr{D}(f)$ - Urbildraum der Funktion $f$ \\
    \item $\mathscr{R}(f)$ - Bildraum der Funktion $f$ \\
    \item $\bar{\Omega}$ - Abschluss einer Menge $\Omega$ \\
    \item $\supp (f)=\left\{ x\in \mathscr{D}(f) : f(x) \neq 0 \right\}$ - Träger der Funktion $f$ \\
    \item $C^{\infty}(\Omega)$ - Menge aller unendlich oft differenzierbarer Funktionen in $\Omega$ \\
    \item $C^{\infty}_{0} (\Omega)\subset C^{\infty}(\Omega)$ - Menge aller Funktionen $\in C^{\infty}$ mit kompaktem Träger in $\Omega$ \\
    \item $\alpha_{m} = \sum_{i=0}^{n}\alpha_{i}$ - Multiindex (bzw. Summe eines Multiindex), $\forall\alpha_i \in \mathbb{N} \textbackslash \{0\}$ \\
\end{itemize}

\newpage

\section{Einführung}

Häufig werden die Euler-Gleichungen aufgrund ihrer einfacheren Handhabung statt der vollständigen Navier-Stokes-Gleichungen verwendet, dazu zählen Anwendungen wie die Untersuchung von Tragflächenprofilen im Flugzeugbau, Untersuchungen zum Blutfluss durch den Körper in der Medizin oder die Modellierung von Rohrströmungen in technischen Bereichen. Da die Euler-Gleichungen ein Spezialfall der Navier-Stokes-Gleichungen sind, sind sie jedoch in guter Näherung nur für laminare, nicht viskose, inkompressible Strömungen anwendbar. Daher finden sich nicht so viele Anwendungsbereiche wie bei den vollständigen Navier-Stokes-Gleichungen. Diese Arbeit soll der Untersuchung des Zusammenhangs zwischen analytischen Lösungsmethoden und der numerischen Berechnung der Euler-Gleichungen gewidmet sein. Im Zuge dessen wird eine eigens erstellte Simulation verwendet, um entsprechend valide numerische Ergebnisse zu erhalten.

\subsection{Was beschreiben die Navier-Stokes-Gleichungen?}

\subsection{Historie}

\subsection{Herausforderungen in der Untersuchung der Navier-Stokes-Gleichungen}

\subsection{Anwendungen in der Praxis}

\section{Physikalische Grundlagen}

\subsection{Strömungslehre}

\subsection{Herleitung der Navier-Stokes-Gleichungen}

\section{Mathematisches Fundament}

\subsection{Grundlagen zu partiellen Differentialgleichungen}

Funktionen, welche durch partielle Differentialgleichungen beschrieben werden sind, im Gegensatz zu solchen, die durch gewöhnliche Differentialgleichungen beschrieben werden, von mehreren Variablen abhängig, sodass nicht nur Ableitungen verschiedener Ordnungen auftreten können, sondern auch derartige, die verschiedene Variablen aufweisen. Daraus entsteht eine enorme Vielfalt, welche unterschiedlichste Lösungsmethoden erfordert. Dieses Kapitel beschränkt sich jedoch auf Formen wie die der Navier-Stokes-Gleichungen, also quasi-lineare Gleichungen zweiter Ordnung.
% erstmal für die Euler-Gleichungen schreiben
Die Euler-Gleichungen beschreiben ein System partieller quasi-linearer Differentialgleichungen zweiter Ordnung. Ein System partieller Differentialgleichungen ist quasi-linear, wenn die Ableitungen höchster Ordnung des Systems linear sind, also nur von den unabhängigen Variablen des Systems sowie (möglicherweise) der Lösungsfunktion selbst abhängig sind. 
\subsubsection{Klassifizierung partieller Differentialgleichungen}
Buch ~\cite[S. 115]{epstein_partial_2017} nennt folgende allgemeine Form partieller Differentialgleichungen zweiter Ordnung:
\begin{align}
\label{zeile1}
    a u_{xx} + 2b u_{xy} + c u_{yy} = d
\end{align}
bzw. im quasi-linearen Fall:
\begin{align*}
    a(x,y,u,u_x,u_y) u_{xx} + 2b(x,y,u,u_x,u_y) u_{xy} + c(x,y,u,u_x,u_y) u_{yy} = d(x,y,u,u_x,u_y).
\end{align*}
Gegebene Anfangsbedingungen der Form 
\begin{align*}
    u=\hat{u}(r), \quad u_x = \hat{u}_x (r), \quad u_y = \hat{u}_y (r)
\end{align*}
(mit dem Kurvenparameter $r$) - welche für die Lösung von Gleichungen dieser Form unerlässlich sind - müssen offensichtlich die Gleichung selbst erfüllen. Außerdem muss nach der Kettenregel und der Bedingung, dass alle Ableitungen an jedem Punkt der gegebenen Kurve $r$ bzw. $\hat{u}(r)$ definiert sein müssen, gelten:
\begin{align}
\label{zeile2}
    \diff{\hat{u}_x}{r} &= u_{xx} \diff{\hat{x}}{r} + u_{xy} \diff{\hat{y}}{r} \\ 
\label{zeile3}
    \diff{\hat{u}_y}{r} &= u_{yy} \diff{\hat{y}}{r} + u_{xy} \diff{\hat{x}}{r}.
\end{align}
Werden nun die Gleichungen ~\eqref{zeile1}, ~\eqref{zeile2} und ~\eqref{zeile3}, welche von den Anfangsbedingungen erfüllt werden müssen, in ein Gleichungssystem kombiniert, ergibt sich:
\begin{align}
\label{lgs_qual}
% Verschönern!!!
    \renewcommand{\arraystretch}{1.15}
    \begin{bmatrix}
        a & 2b & c \\
        \diff{\hat{x}}{r} & \diff{\hat{y}}{r} & 0 \\
        0 &  \diff{\hat{x}}{r} & \diff{\hat{y}}{r} \\
    \end{bmatrix}
    \begin{bmatrix}
        u_{xx} \\
        u_{xy} \\
        u_{yy} \\
    \end{bmatrix}
    =
    \begin{bmatrix}
        d \\
        \diff{\hat{u}_x}{r} \\
        \diff{\hat{u}_y}{r}
    \end{bmatrix}
\end{align}
damit dieses Gleichungssystem eindeutig lösbar ist, muss die Determinante der Koeffizientenmatrix ungleich null sein. Nach der \emph{Regel von Sarrus} ergibt sich für Gleichung ~\eqref{lgs_qual}:
\begin{align}
\label{determinante}
\determ{A} = a\cdot \diff{\hat{y}}{r} \cdot \diff{\hat{y}}{r} + 0 + c \cdot \diff{\hat{x}}{r}\cdot \diff{\hat{x}}{r} - 2b \cdot \diff{\hat{x}}{r} \cdot \diff{\hat{y}}{r} - 0 - 0.
\end{align}
Wäre das Gleichungssystem an jeder Position auf der Anfangskurve $r(x,y)$ eindeutig lösbar, sind diese Anfangsbedingungen nicht ausreichend, um die partielle Differentialgleichung zu lösen, da sie an keiner Stelle vorschreiben, wie sich die Lösungsfunktion außerhalb dieser gegebenen Kurve verhalten. Es lässt sich zeigen, \begin{comment}
    ~\cite{Epstein}
\end{comment}
dass gerade an den Stellen, an denen das Gleichungssystem nicht eindeutig lösbar ist, Charakteristiken (eine Definition soll hier nicht genannt werden) aufweist. Diese sind essenziell für die Lösungstheorie partieller Differentialgleichungen und bestimmen die Art und Weise partielle Differentialgleichungen zu lösen. Setzen wir ~\eqref{determinante} gleich null, und multiplizieren beide Seiten mit $\left(\diff{r}{\hat{x}}\right)^2$, erhalten wir eine quadratische Gleichung, deren Lösungen mit der \emph{pq-Formel} gefunden werden können:
\begin{align}
\label{quad_gleichung_det}
    \lr{(}{)}{\diff{\hat{y}}{\hat{x}}}_{1,2} = \frac{b}{a} \pm \sqrt{\lr{(}{)}{\frac{b}{a}}^{2} - \frac{c}{a}}.
\end{align}
Für die Diskriminante $D$ der Gleichung können folgende Fälle eintreten:
\begin{align}
    x_{1,2} = 
    \begin{cases}
        D>0: \quad &x_1 \neq x_2  \land x_1, x_2 \in \mathbb{R} \\
        D=0: \quad &x_1 = x_2 \land x_1, x_2 \in \mathbb{R} \\
        D<0: \quad &x_1 \neq x_2 \land x_1, x_2 \in \mathbb{C} \\
    \end{cases}
\end{align}
\begin{Definition}
    Eine partielle Differentialgleichung zweiter Ordnung ist: \\
    \textbf{hyperbolisch}, wenn es zwei unterschiedliche Lösungen ($\in \mathbb{R}$) der Gleichung ~\eqref{quad_gleichung_det} gibt, \\
    \textbf{parabolisch}, wenn es eine Lösung ($\in \mathbb{R}$) gibt, \\
    \textbf{elliptisch}, wenn es zwei unterschiedliche Lösungen ($\in \mathbb{C}$) gibt.
\end{Definition}

\subsection{Sobolew-Räume}

Zunächst soll das Lebesgue-Maß sowie das Lebesgue-Integral erklärt werden. 
\begin{Definition}
    Das Lebesgue-Maß $\mu(E)$ einer Menge $E$, welche als $n$-dimensionales Hyperrechteck ($E\subset \mathbb{R}^n$) identifiziert werden kann, ist definiert als:
    \begin{align*}
        \mu(E)= \prod_{i=1}^{n}{(b_i - a_i)}
    \end{align*}
    genau dann wenn:
    \begin{align*}
        E=[a_1 , b_1] \times [a_2 , b_2] \times \ldots \times [a_n , b_n]
    \end{align*}
\end{Definition}
Das Lebesgue-Maß ordnet damit anschaulich einem $n$-dimensionalen Hyperrechteck ein Volumen zu.
Das Lebesgue-Integral zerlegt im Gegensatz zum Riemann-Integral nicht das Integrationsintervall des Urbildraums der Funktion $f:\mathbb{R} \ni x \mapsto f(x)\in \mathbb{R}$ in Teilintervalle der Größe $\Delta x_i \to 0$, sondern das Intervall des Bildraums der Funktion $f$.
\begin{Definition}
    Sei $c_i$ eine Zerlegung des Intervalls $[\min(f),\max(f)]$, sodass $\bigcup_{i\in \mathbb{N}\cup \{\infty\}}c_i = \mathscr{R}(f)$, dann ist das Lebesgue-Integral der Funktion $f(x)$ definiert als:
    \begin{align*}
        \sum_{i}{c_i \mu(A_i)} = \int{f(x) \dfr{\mu}}
    \end{align*}
\end{Definition}

\subsection{Existenz- und Eindeutigkeitssätze für PDEs}

Um die Eindeutigkeit bzw. grundlegende Existenz von Lösungen für partielle Differentialgleichungen zu beweisen, ist es in der physikalischen Realität - im Gegensatz zu den idealisierten Bedingungen der Mathematik - häufig sinnvoll, statt vollständige Lösungen für genannte Gleichungen zu finden, schwache Lösungen zu berechnen.
\begin{Definition}
    Sei die $n$-dimensionale partielle Differentialgleichung
    \begin{align}
    \label{PDE_Existenz}
        a u_{xx} + 2b u_{xy} + c u_{yy} = d
    \end{align}
    gegeben. Eine Funktion $u\in \mathbb{R}$ ist eine \emph{schwache Lösung} von Gleichung~\eqref{PDE_Existenz} auf der Menge $A$, wenn für $\forall \varphi \in C_0 ^{\infty} (A)$ folgende Bedingung erfüllt ist:
    \begin{align*}
        \int_{A}{\varphi (a u_{xx} + 2b u_{xy} + c u_{yy}) \dfr x} = \int_{A}{\varphi d \dfr x}
    \end{align*}
    bzw. mit der Umformung in Abschnitt~\ref{sec:anhang_rechnung_1}: % HIER VERBESSERN, dass der gesamte Unterabschnitt ausgeschrieben wird
    \begin{align*}
        % HIER UMFORMUNG EINFÜGEN
    \end{align*}
\end{Definition}
Mithilfe dieser schwachen Lösungsvorgabe lässt sich einsehen, dass eine entsprechende Lösung im Sobolev-Raum liegen muss.

\subsection{Approximationsmethoden}

\subsubsection{Conjugate-Gradient-Methode}

Die Conjugate-Gradient-Methode erweist sich als sehr nützlich bei der Lösung linearer Gleichungssysteme der Form
\begin{align}
    Ax=b \label{LGS},
\end{align}
wobei $b$ ein konstanter Ergebnisvektor ist, $x$ ein Variablen-Tupel, welches mit Hilfe der Conjugate-Gradient-Methode ermittelt werden soll und $A$ eine positiv definite Koeffizientenmatrix.
\begin{Definition}
    Eine Matrix $A$ ist positiv definit, wenn alle ihre Eigenwerte positiv sind.
\end{Definition}
Dabei bedient sich die Conjugate-Gradient-Methode nicht direkt diesem Gleichungssystem, sondern minimiert eine Funktion in der sogenannten quadratischen Form:
\begin{align}
    f(x)&=\frac{1}{2}x^T A x - x^T b \qquad (x \in \mathbb{R}^n). \label{quad_form}
\end{align}
Der Gradient dieser Funktion $\nabla f(x)$ ist gegeben durch:
\begin{align*}
    \nabla f(x) = \frac{1}{2} A^T x + \frac{1}{2}Ax - b,
\end{align*}
wobei der rechte Teil dieser Gleichung äquivalent zu Gleichung \eqref{LGS} ist, wenn die Matrix $A$ symmetrisch ist. Gleichung \eqref{quad_form} hat die angenehme Eigenschaft, nur ein Extremum zu besitzen, wobei, dadurch dass $A$ positiv definit ist, dieses Extremum ein Minimum ist. Ist also $\nabla f(x)=0$, so ist das Minimum gefunden. Ist also durch die Conjugate-Gradient-Methode ein Minimum der genannten Gleichung gefunden, so gilt:
\begin{align*}
    \nabla f(x)=0=Ax-b \quad \therefore \, Ax=b %KORRREKTUR: ,
\end{align*}
und das Gleichungssystem ist gelöst.
Die Conjugate-Gradient-Methode basiert dabei auf dem Prinzip der Conjugate-Directions, wobei Suchrichtungen geschaffen werden, die jeweils $A$-orthogonal sind.
\begin{Definition}
    Zwei Vektoren $x$, $y$ sind $A$-orthogonal, wenn gilt:
    \begin{align*}
        x\cdot Ay = 0,
    \end{align*}
    dabei ist $\cdot$ ein geeignetes Skalarprodukt.
\end{Definition}
Es lässt sich nämlich zeigen, dass, wenn die Suchrichtungen $A$-orthogonal sind, das Verfahren genau dann in $n$-Schritten konvergiert, wenn in jeder dieser Suchrichtungen die Gleichung \eqref{quad_form} minimiert wird. Beide dieser Bedingungen werden mit dem folgenden Verfahren erfüllt: 
\begin{enumerate}
    \item Bestimmung des Rests ($b-Ax_0$), dieser ist die erste Suchrichtung \\
    \item Bestimmung des Koeffizienten $\alpha$, welcher die Gleichung entlang der Suchrichtung minimiert ($\min(f(x_i + \alpha d_i))$) \\
    \item Setze $x_{i+1} = x_i + \alpha d_i$ als neue Lösung \\
    \item Setze den Rest als $r_{i+1} = r_i - \alpha Ad_i$ \\
    \item Bestimmung des Koeffizienten $\beta$, welcher die neue Suchrichtung ergibt: $\beta = \frac{r_{i+1}^T r_{i+1}}{r_i^T r_i}$ \\
    \item Setzen der neuen Suchrichtung: $d_{i+1} = r_{i+1} + \beta d_i$ \\
\end{enumerate}
Im Falle der Conjugate-Gradient-Methode notiert dabei: $d$ die Suchrichtung des Algorithmus, $r$ den Rest (Abstand zur korrekten Lösung) sowie $\alpha$ und $\beta$ entsprechende Koeffizienten.

% Ergänzung der Bedeutungen der einzelnen Symbole

\subsubsection{Diskretisierung von Gebieten}

In der numerischen Lösung von partiellen Differentialgleichungen, welche auf einem Gebiet $\Omega$ definiert sind, wird die Diskretisierung durch \glqq Meshes\grqq (Deutsch: Gitter/Netz) angewendet, welche unendlich viele Punkte dieses Gebiets durch diskrete Punkte ersetzt, an denen entsprechende Zahlenwerte gegeben sind, sodass nur endlich viele Zahlenwerte gespeichert werden müssen. Diese Punkte werden als Schnittpunkte entsprechender Linien konstanter Koordinaten angegeben (siehe Abbildung~\ref{fig:meshs}). Diese Vereinfachung ermöglicht es, Ableitungen als finite Differenzen zwischen benachbarten Knoten approximativ zu berechnen. Neben orthogonalen Meshes existieren auch krummlinige, welche die Annäherung um bestimmte Formen (wie z.B. Kreise oder Zylinder) durch ein angepasstes krummliniges Koordinatensystem ermöglicht. Auf solche Methoden soll nicht eingegangen werden, da eine Anwendung dieser in der Simulation mit finiten Differenzen eigene komplexe Probleme mit sich bringt.
\begin{figure}
    \centering
    \def\svgwidth{0.75\textwidth}
    \input{Abbildungen/Abbildung_Diskretisierung.pdf_tex}
    \caption{Diskretisierung des zweidimensionalen Gebietes $\Omega$}
    \label{fig:meshs}
\end{figure}

\subsubsection{Annäherung von Ableitungen}

Da im Vergleich zu analytischen Methoden aufgrund der physischen Beschränkungen von Computern eine unendlich kleine Differenz nicht zu berechnen ist, werden numerische Methoden verwendet, um solche Differenzen möglichst effizient bzw. präzise anzunähern.
Grundsätzlich ließe sich eine Ableitung folgendermaßen annähern:
\begin{align*}
    \left.\pd{f(x,y)}{x}\right\vert_{x_0 , y_0} = \frac{f(x_0 +\Delta x,y_0)-f(x_0,y_0)}{\Delta x} + \mathcal{O}(\Delta x)
\end{align*}
diese Art der Annäherung ist zwar einfach, jedoch nicht effizient, da sie Fehler der Größenordnung $x$ enthält. Mithilfe der \emph{Taylorreihenentwicklung} lassen sich effizientere Methoden entwickeln. Wird das Verfahren aus ~\cite[S. ~51f.]{lecheler_computational_2022} auf Restterme der Größenordnung $\mathcal{O}(\Delta x^4)$ angewendet, ergibt sich folgende Rechnung:
\begin{equation}
\begin{align*}
\label{naeherung4}
    \left. \pd{f(x,y)}{x} \right\vert_{x_0,y_0} &= \frac{f(x_0-2\Delta x,y_0) - 8 f(x_0 -\Delta x,y_0)}{12\Delta x} \\
    &+ \frac{8f(x_0+\Delta x, y_0) - f(x_0 + 2\Delta x, y_0)}{12\Delta x} + \mathcal{O}(\Delta x^4),
\end{align*}
\end{equation}
da ($f^{(n)}$ ist als Ableitung nach $x$ zu verstehen)
\begin{align*}
    q_1 &= f(x_0 -2\Delta x) = f(x_0) -2\left.f^{(1)}\right\vert_{x_0,y_0} \Delta x + 2 \left.f^{(2)}\right\vert_{x_0,y_0} \Delta x^2 - \frac{8}{6} \left.f^{(3)}\right\vert_{x_0,y_0} \Delta x^3 \\
    &+ \frac{16}{24}\left.f^{(4)}\right\vert_{x_0,y_0} \Delta x^4 + \mathcal{O}(\Delta x ^5) \\
    q_2 &= -8f(x_0 -\Delta x) = -8f(x_0)+8\left.f^{(1)}\right\vert_{x_0,y_0} \Delta x -4 \left.f^{(2)}\right\vert_{x_0,y_0} \Delta x^2 + \frac{8}{6} \left.f^{(3)}\right\vert_{x_0,y_0} \Delta x^3 \\
    &- \frac{8}{24}\left.f^{(4)}\right\vert_{x_0,y_0} \Delta x ^4+ \mathcal{O}(\Delta x ^5) \\
    q_3 &= 8f(x_0 +\Delta x) = 8f(x_0) +8\left.f^{(1)}\right\vert_{x_0,y_0} \Delta x +4 \left.f^{(2)}\right\vert_{x_0,y_0} \Delta x^2 + \frac{8}{6} \left.f^{(3)}\right\vert_{x_0,y_0} \Delta x^3 \\
    &+ \frac{8}{24}\left.f^{(4)}\right\vert_{x_0,y_0} \Delta x ^4+ \mathcal{O}(\Delta x ^5) \\
    q_4 &= -f(x_0 +2\Delta x) = -f(x_0) -2\left.f^{(1)}\right\vert_{x_0,y_0} \Delta x -2 \left.f^{(2)}\right\vert_{x_0,y_0} \Delta x^2 - \frac{8}{6} \left.f^{(3)}\right\vert_{x_0,y_0} \Delta x^3 \\
    &- \frac{16}{24}\left.f^{(4)}\right\vert_{x_0,y_0} \Delta x^4 + \mathcal{O}(\Delta x ^5) \\
    \Rightarrow \sum_{b=1}^{4} q_b &= f(x_0-2\Delta x) - 8f(x_0 - \Delta x) + 8f(x_0 + \Delta x) - f(x_0 + 2\Delta x) \\
    &= 0 + 12 \left.f^{(1)}\right\vert_{x_0 , y_0} \Delta x + 0 + 0 + 0 + \mathcal{O}(\Delta x^5) \\
    \Rightarrow \frac{\sum_{b=1}^{4} q_b}{12\Delta x} &= \left.f^{(1)}\right\vert_{x_0 , y_0} + \mathcal{O}(\Delta x^4). \\
\end{align*}
Da diese nur Fehler der Ordnung $\mathcal{O}(\Delta x^4)$ und höher enthält, soll diese Art der Annäherung für die Simulation verwendet werden.
Auch zweite (gemischte) Ableitungen lassen sich mit Gleichung~\ref{naeherung4} bestimmen, indem als Funktion die erste Ableitung eingesetzt wird und diese ebenfalls angenähert wird, sodass sich die Gleichung:
\begin{align*}
    \pdsu{f}{x}{y} &= \frac{\left( \right)}{} \\ % HIER NOCH GLEICHUNG EINFÜGEN
\end{align*}
ergibt.
Diese Annäherungen sind wichtiger Bestandteil der Simulation, welche in~\autoref{sec:sim} erläutert wird.

% Gegebenenfalls bei Simulation Gleichungen plattformunabhängig verwenden, Ergebnisse in R darstellen

\subsection{Lösung für Spezialfälle der Navier-Stokes-Gleichungen}

\section{Untersuchungsmethode}

\subsection{Erläuterung der Simulation}

\label{sec:sim}

\subsection{Annahmen mathematischer/ physikalischer Natur}

\section{Darstellung der Forschungsergebnisse}

\section{Diskussion}

\section{Fazit}

\section{Anhang}

\subsection{zusätzliche Rechnungen}

\label{sec:rechnungen}

\subsubsection*{§1}

\label{sec:anhang_rechnung_1}

\begin{align*}
    u
    \int_{A}{\varphi (a u_{xx} + 2b u_{xy} + c u_{yy}) \dfr x} &= 
\end{align*}

\subsection{Quelldateien}

\section{Abbildungsverzeichnis}

\printbibliography

\end{document}