\documentclass[size=12pt]{report}
\usepackage[ngerman]{babel}
\usepackage[utf8]{inputenc}
\usepackage[T1]{fontenc} 
\usepackage{csquotes}
\usepackage{blindtext}
\usepackage{array, multicol, multirow}
\usepackage{lipsum}
\usepackage{tabularx}

\usepackage{amsmath}
\usepackage{amssymb}
\usepackage{siunitx}
\sisetup{locale = DE}
\usepackage{mathtools}
\usepackage{amsthm}
\usepackage{graphicx}
\usepackage{unicode-math}
\usepackage{braket}

\usepackage{mathrsfs}

\newtheorem{theoremvar}{Theorem}
\newtheorem{definition}{Definition}

\title{Funktionalanalysis}

\begin{document}

\begin{definition}
    Der Urbildraum eines Operators $A:U\mapsto I$ ist definiert als
    \begin{align*}
        \mathscr{D}(A):= \{u\in U: Au\neq \{\}\}
    \end{align*}
\end{definition}

\begin{definition}
    Der Bildraum eines Operators $A:U\mapsto I$ ist definiert als 
    \begin{align*}
        \mathscr{R}(A):= \{u\in \mathscr{D}(A): Au\}
    \end{align*}
\end{definition}

\begin{definition}
    Der Abschluss (closure) $\Bar{M}$ einer Menge $M$ ist definiert als\footnote{Die Folge der $\{x_n\}$ wird als Cauchy-Folge vorausgesetzt.}
    \begin{align*}
        \forall \{x_n\}_{\infty} \in M \rightarrow x \Rightarrow x\in \Bar{M}
    \end{align*}
\end{definition}

\begin{theoremvar}
Sei $A$ ein dicht definierter, geschlossener linearer Operator $A:\mathscr{D}(A)\subset \emph{H}_1 \mapsto \emph{H}_2$. Dann ist $\mathscr{D}(A^*)$ dicht in $\emph{H}_2$ und es gilt $A^{**}=A$.
\end{theoremvar}
\\
\emph{Beweis:}
Da zwischen den Mengen $\mathscr{D}(A)$ und $\emph{H}_2$ eine lineare Abbildung besteht, muss es eine Bijektion zwischen diesen Mengen geben. Nach der Definition des adjungierten Operators $A^*:\mathscr{D}(A^*)\subset \emph{H}^*_2 \mapsto \mathscr{R}(A^*) \subset \emph{H}^*_1 \text{ mit } \braket{A^*v;\, u} = \braket{v;\, Au},\, \forall u\in H_1, v\in H_2^*$ muss es zu einem Funktional (für $\mathbb{K}=\mathbb{R}$) aus $\emph{H}_1^*$ ($\in \mathscr{R}(A^*)$) und einem Element aus $\emph{H}_1$ ein Element aus $H_2^*$ geben, welches einem $Au \in H_2$ denselben Funktionswert zuordnet.
Dadurch, dass der Operator bijektiv ist, lässt sich die Schreibweise durch die Äquivalenz eines Hilbertraumes mit seinem Dualraum ändern und es gilt $\braket{v;\, Au}=\braket{Au;\, v}$, sodass obige Aussage zu $\braket{A^*v;\, u}=\braket{Au;\, v}$ wird. Nach dem Riesz-Repräsentations-Theorem lässt sich diese Aussage weiterhin umwandeln zu $(w_{H_1};\, u)_{H_1}=(w_{H^*_2};\, v)_{H^*_2},\, (u\in H_1, v\in H_2^*)$. Da beide Seiten der Gleichung jeweils eine bijektive Abbildung darstellen, muss es zu jedem $u\in H_1$ ein eindeutiges Element $\in H_1$ geben. Da dies auf der Gegenseite ähnlich ist, muss es zu jedem - als Skalarprodukt in $H_1$ darstellbarem - Funktionswert den selben in $H_2^*$ geben. Daraus resultiert, dass beide Mengen gleichmächtig sein müssen und, da $A^*$ mit $A$ als linear und damit bijektiv gegeben war, impliziert dies, dass $D(A^*)$ dicht in $H_2^*$ sein muss (da $H_2^*=H_2$ gleichmächtig wie $H_1$ ist).
Wendet man dieses Argument auch auf den adjungierten Operator $A^*$ an, so muss $\forall (a\in \mathscr{D}(A^{**})\Rightarrow a\in \mathscr{D}(A))$ gelten (sowie für die Reichweite)\footnote{Die Linearität des adjungiertadjungierten Operators zu $A$ folgt aus der vorherigen Anmerkung.}.
\end{document}
